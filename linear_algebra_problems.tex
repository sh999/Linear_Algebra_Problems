

\documentclass{book}
\usepackage{amsmath}
\usepackage{amssymb}
\title{Linear Algebra Problems and Working Solutions}
\date{}
\renewcommand{\vec}[1]{\mathbf{#1}}
\begin{document}
	\maketitle
	\chapter{Vector Spaces}
	\underline{Topics:}
	\begin{enumerate}
		\item Basic vector operations
		\item Vector space
		\item Basic matrix manipulations
	\end{enumerate}	
	\underline{Problems:}
	
	\noindent\textbf{(1.1 Treil)}
	Let $\vec{x} = (1,2,3)^T$,
	$\vec{y} = (y_1,y_2,y_3)^T$, 
	$\vec{z} = (4,2,1)^T.$ 
	\\Compute $2\vec{x}, 3\vec{y}, \vec{x}+2\vec{y}-3\vec{z}.$  	
	\\\textit{Answer:}
	
	$
	2\vec{x}=2 
	\begin{bmatrix}
	1\\2\\3
	\end{bmatrix}
	= \begin{bmatrix}
	2\\4\\6
	\end{bmatrix}
	$,\\\\

	$
	2\vec{y}=2 
	\begin{bmatrix}
	y_1\\y_2\\y_3
	\end{bmatrix}
	= \begin{bmatrix}
	3\cdot y_1\\3\cdot y_2\\3\cdot y_3
	\end{bmatrix}
	$,\\\\

	$
	\vec{x}+2\vec{y}-3\vec{z}= 
	\begin{bmatrix}
	1\\2\\3
	\end{bmatrix}+
	\begin{bmatrix}
	2\cdot y_1\\2\cdot y_2\\2\cdot y_3
	\end{bmatrix}-
	\begin{bmatrix}
	12\\6\\3
	\end{bmatrix}=
	\begin{bmatrix}
	2y_1-11\\2y_2-4\\2y_3
	\end{bmatrix}
	\\\square $\\\\

	
	\noindent\textbf{(1.8 Treil)}
	Prove that for any vector $\vec{v}$ its additive inverse is $-\vec{v}$ is given by $(-1)\vec{v}$.
	\\\textit{Answer:}
	
	The additive inverse axiom for vector spaces says that $\forall \vec{v} \in V, \exists \vec{w} \in V$ such that $\vec{v}+\vec{w}=\vec{0}.$
	If $\vec{v}$ is in field $\mathbb{R}^n$,
	
	$\vec{v}=(v_1,v_2,...,v_n).$
	  
	$(-1)\vec{v}=(-1)(v_1,v_2,...,v_n)=(-v_1,-v_2,...,-v_n)$
	$\vec{v}+(-1)\vec{v}=(0,0,...0)=\vec{0}$.
	Thus,$(-1)\vec{v}$ is the additive inverse of $\vec{v}.\\
	\square$\\\\
	
		
	
	
\end{document}\leftthreetimes 